\documentclass[11pt]{aghdpl}
% \documentclass[en,11pt]{aghdpl}  % praca w języku angielskim
\usepackage[polish]{babel}
%\usepackage[english]{babel}
\usepackage[utf8]{inputenc}
\usepackage[T1]{fontenc}

% dodatkowe pakiety
\usepackage{enumerate}
\usepackage{listings}
\lstloadlanguages{TeX}

%kolorowanie linków + działanie linków
\usepackage[colorlinks=true,urlcolor=blue,linkcolor=red,citecolor=green]{hyperref}



\usepackage{listings}
\usepackage{xcolor}
\definecolor{listinggray}{gray}{0.9}
\definecolor{lbcolor}{rgb}{0.9,0.9,0.9}
\lstset{
    backgroundcolor=\color{lbcolor},
    tabsize=4,
  language=C++,
  captionpos=b,
  tabsize=3,
  frame=lines,
  numbers=left,
  numberstyle=\tiny,
  numbersep=5pt,
  breaklines=true,
  showstringspaces=false,
  basicstyle=\footnotesize,
%  identifierstyle=\color{magenta},
  keywordstyle=\color[rgb]{0,0,1},
  commentstyle=\color{Darkgreen},
  stringstyle=\color{red}
  }
\lstset{
  literate={ą}{{\k{a}}}1
           {ć}{{\'c}}1
           {ę}{{\k{e}}}1
           {ó}{{\'o}}1
           {ń}{{\'n}}1
           {ł}{{\l{}}}1
           {ś}{{\'s}}1
           {ź}{{\'z}}1
           {ż}{{\.z}}1
           {Ą}{{\k{A}}}1
           {Ć}{{\'C}}1
           {Ę}{{\k{E}}}1
           {Ó}{{\'O}}1
           {Ń}{{\'N}}1
           {Ł}{{\L{}}}1
           {Ś}{{\'S}}1
           {Ź}{{\'Z}}1
           {Ż}{{\.Z}}1
}



%\usepackage[pdftex]{graphicx}
\usepackage{tikz}
%---------------------------------------------------------------------------

\author{Żaneta Błaszczuk, Rafał Kozik, Filip Kubicz, Jakub Nowak, Jakub Porębski}
\shortauthor{Ż. Błaszczuk, R. Kozik, F. Kubicz, J. Nowak, J. Porębski}

\titlePL{Generator liczb losowych}
\titleEN{Random number generator}

\shorttitlePL{Generator liczb losowych} % skrócona wersja tytułu jeśli jest bardzo długi
%\shorttitleEN{Thesis in \LaTeX}

\thesistype{Modelowanie układów fizycznych i biologicznych}
%\thesistype{Master of Science Thesis}

\supervisor{dr inż. Ireneusz Wochlik}

\degreeprogramme{Automatyka i Robotyka}
%\degreeprogramme{Computer Science}

\date{2014}

\department{Katedra Automatyki}
%\department{Department of Applied Computer Science}

\faculty{Wydział Elektrotechniki, Automatyki,\protect\\[-1mm] Informatyki i Inżynierii Biomedycznej}
%\faculty{Faculty of Electrical Engineering, Automatics, Computer Science and Biomedical Engineering}

\setlength{\cftsecnumwidth}{10mm}

\begin{document}
\titlepages
\section{Wstęp}
\subsection{Cel zastosowania generatorów liczb pseudo losowych}
Liczby pseudo losowe przedewszystkim wykorzystuje się w 3 dziedzinach:
\begin{enumerate}
\item Olbiczeniach numerycznych – wykorzystywane w metodach obliczeniowych, które nie wymagaja dużej ilości liczb o danym rozkładzie.
\item Kryptografia – wykorzystywane do generowania kluczy prywatnych
\item Złudzenie losowości w grach – wykorzystując generatory gracz ma poczucie przewidywalności i nie powtarzalności.  
\end{enumerate}
\subsection{Liczby pseudo losowe}
Liczbami pseudo losowymi nazywamy liczby wykazujące cechy liczb prawdziwie losowych uzyskanych poprzez działanie algorytmu. Zaletą korzystania z takich liczb jest możliwość szybkiego pozyskiwania rezultatów, ograniczona mocą obliczeniową komputera.
\subsection{Generatory liczb pseudo losowych}
Liczby pozyskiwane są z odpowiednich algorytmów, obliczających kolejną liczbę na podstawie poprzedniego wyniku. W związku z tym konieczne jest podanie pierwszej liczby. W praktyce generatory wykorzystują jako pierwszą liczbę aktualny czas. 
\section{Testowanie generatorów}
\subsection{Cel testowania generatorów}
Generatory testowane są przedewszystkim by odrzucić te z nich, które dają wyniki nie losowe. Najważniejszym założeniem jest, by nie dopuścić do przepuszczenia złego generatora, nawet kosztem pomyłkowego odrzucenia działającego. W dzisiejszych czasach mamy do dyspozycji bardzo wiele różnych generatorów. Wybierając generator źle działający możemy mieć doczynienia z bardzo dotkliwymi stutkami dla użytkownika. Dlatego też zostały sformuowane „podstawowe cechy dobrego generatora”:
\begin{enumerate}
\item Jednorodność – prawdopodobieństwo wystopienia 1 lub 0 wynosi 0,5 w każdym punkcie generowaniego ciągu bitów.
\item Skalowalność - każdy podciąg ciągu, który zakończył się pozywytnym testem, również powinien zakończyć się tym samym pozytywnym testem.
\item Zgodność – zachowanie generatora powinno dawać podobne rezultaty niezależnie od początkowej wartości.
\end{enumerate}
\subsection{Analiza rezultatów}
Testy, na których się oparłem pochądzą z pakietu DieHard, oraz z jego następcy DieHarder, uznawanego za jednego z najlepszych pakietów. Rozpowszechniony jest na licencji GPL, co pozwala na zgłębienie kodu oraz modyfikacji, jeśli zajdzie taka potrzeba.Atutem DieHard jest też przejżysty i powtarzalny sposób prezentacji testów. DieHard wynik przedstawia nam w postaci p-wartość, która informuje nas o odchyleniu rozkładu od wartości oczekiwanej. Dobre generatory w testach zbliżają się ze swoją p-wartość do 1. Natomiast wartością graniczną, przy której generator przestaje być losowy jest p-wartość~=~0,05. 
Ten sam test powtarza się zazwyczaj dla danego generatora n razy (zazwyczaj n = 100), w celu uniknięcia zdarzenia, że generator raz przeszedł dany test. Następnie otrzymane p-wartości przepuszcza się przez test Kolmogorova-Smirnova, by sprawdzić czy ich wrtości mają rozkład równomierny.
\section{Przykładowe testy}
\subsection{Wybrane generatory do testów}
\subsubsection{KISS – (Keep Is Simple Stupid)}
Generator złożony z trzech prostych generatorów. Ideą jego jest bycie szybkim i prostym. KISS składa się z:
\begin{equation}
x(n) = a \cdot x (n-1)+1 \; mod \; 2^{32}
\end{equation}
\begin{equation}
y(n)=y(n-1)(I+L^{13})(I+R^{17})(I+L^5)
\end{equation}
\begin{equation}
z(n)=2 \cdot z(n-1)+z(n-2) +carry \; mod \; 2^{32}
\end{equation}
Generator ten nadaje się do programowania w asemblerze, gdzie trwa około 200 nanosekund przy procesorze Pentium 120.
\subsubsection{Generator System Microsoft Fortran}
Generator oparty na 32 bitach opisany wzorem:
\begin{equation}
X ( n ) = 48271 \cdot x (n- 1 )\; mod\; 2 ^ {31}-1,
\end{equation}
zbliżony do Lehmer, charakteryzuje się jednak tym, że zaproponowano w nim lepszy mnożnik. 
\subsubsection{Odwrócony generator}
Opisany przez Einchenauer-Herrmann. Obliczenia zajmują dużo więcej czasu. Sprawdza się gorzej, niż klasyczny RNG mod $2^{32}$
\subsection{Wybrane testy}
\subsubsection{Test urodzin}
Polega on na „paradoksie urodzin”, czyli prawdopodobieństwie, że w grópie osób, znajdą się osby które mają urodziny tego samego dnia. Testowana grupa ma 512 osób, a dzień określa 24 bitowa liczba. Rozkład powinien być podobny do rozkładu Poissona.
\subsubsection{Minimum Distance test}
Wybiera się 8000 losowych punktów w kwadracie o boku 10000. Oblicza się d (minimalna odległość pomiędzy $\frac{n^2 -n}{2}$ parami punktów). Rozkład powinien przedstawiać rozkład równomierny.
\subsubsection{Sums test}
Polega na przekształceniu liczby losowych całkowitych na liczby zmiennoprzecinkowe. Następnie wylicza się sumy pokrywających się podciągów składających się ze 100 elementów. Otrzymany ciąg sum powinienn mieć rozkład normalny.
\section{Testowanie}
\subsection{Test urodzin}
\begin{center}
\begin{tabular}{ccc}
generator & Finalna p-wartość & Wynik \\ \hline
KISS &  0.934361  & PASSED \\
System Microsoft Fortran & 0.995980 & PASSED \\
Odwrócony generator & 0.804086 & PASSED \\
\end{tabular}
\end{center}
Wszystkie generatory zdały test.
\subsection{Minimum Distance test}
\begin{center}
\begin{tabular}{ccc}
generator & Finalna p-wartość & Wynik \\ \hline
KISS &  0.309341  & PASSED \\
System Microsoft Fortran & 0.772850 & PASSED \\
Odwrócony generator & 0.046785 & FAILED \\
\end{tabular}
\end{center}
Odwrócony generator nie zdał testu. Z testem dobrze poradził sobie natomiast SMF.
\subsection{Sums test}
\begin{center}
\begin{tabular}{ccc}
generator & Finalna p-wartość & Wynik \\ \hline
KISS & 0.593360 & PASSED \\
System Microsoft Fortran & 0.808943 & PASSED \\
Odwrócony generator & 0.315619  & PASSED \\
\end{tabular}
\end{center}
Wszystkie generatory zdały test, chodź najlepiej poradził sobie SMF.
\subsection{Podsumowanie}
Z wybranych generatorów najlepiej sprawuje się generator System Microsoft Fortran, który, zakończył każdy test pozytywnie i był w tych testach najbardziej zbliżony do 1.
\end{document}