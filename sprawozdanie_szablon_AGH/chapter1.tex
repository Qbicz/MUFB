\section{Równania ruchu}
\begin{figure}[h!]
\centering
\label{Schemat}
\setlength{\unitlength}{2mm}
\begin{picture}(50,50)(0,0)

\put(25,50){\vector(1,0){3}}
\put(25,50){\vector(0,-1){3}}
\put(27,49){\makebox(0,0){$x$}}
\put(24,48){\makebox(0,0){$y$}}

\put(20,50){\line(1,0){10}}
\multiput(25,48)(0,-2){7}{\line(0,1){1}}
\put(25,50){\line(1,-3){5}}
\put(30,35){\circle*{2}}

\put(27,38){\makebox(0,0){$\varphi_1$}}
\put(30,40){\makebox(0,0){$l_1$}}
\put(33,35){\makebox(0,0){$m_1$}}

\multiput(30,33)(0,-2){7}{\line(0,1){1}}
\put(30,35){\line(1,-5){4}}
\put(34,15){\circle*{2}}

\put(31.5,22){\makebox(0,0){$\varphi_2$}}
\put(33,25){\makebox(0,0){$l_2$}}
\put(37,15){\makebox(0,0){$m_2$}}

\end{picture}
\caption{Schemat wachadła}
\end{figure}
Równania ruchu można wyznaczyć korzystając z równania Lagrangea-Eurela. Energia potencjalna układu jest równa:
\begin{equation}
	V=m_1gl_1 \cos(\varphi_1)+m_2g(l_1 \cos(\varphi_1)+l_2 \cos(\varphi_2)
\end{equation}
Natomiast energia kinetyczna układu wynosi:
\begin{equation}
	K=\frac{1}{2}m_1v_1^2+\frac{1}{2}m_2v_2^2
\end{equation}
Aby uprościć równania, przyjmujemy:
\begin{equation}
	A = \frac{m_1}{m_2} \qquad B = \frac{l_2}{l_1} \qquad C = \frac{g}{l_1}
\end{equation}
Ruch wachadła opisuje równania:
\begin{equation}
	\dot{\varphi}_1 = \omega_1
\end{equation}
\begin{equation}
	\dot{\omega}_1=-\frac{\sin(\varphi_1-\varphi_2)(B\omega_2^2+\omega_1^2cos(\varphi_1-\varphi_2))+C((A+1)\sin(\varphi_1)-
	\sin(\varphi_2)cos(\varphi_1-\varphi_2))}{A+\sin^2(\varphi_1-\varphi_2)}
\end{equation}
\begin{equation}
	\dot{\varphi}_2 = \omega_2
\end{equation}
\begin{equation}
	\dot{\omega}_2 = \frac{(A+1)(\omega_1^2\sin(\varphi_1-\varphi_2)-C\sin(p2))+\cos(\varphi_1-\varphi_2)((B\omega_2^2 \sin(\varphi_1-\varphi_2))+C(A+1)\sin(\varphi_1))}{B(A+\sin^2(\varphi_1-\varphi_2)}
\end{equation}

