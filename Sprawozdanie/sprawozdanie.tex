\documentclass[11pt,a4paper]{article}
\usepackage[utf8]{inputenc}
\usepackage[T1]{fontenc}
\usepackage{amsmath}
\usepackage{amsfonts}
\usepackage{amssymb}
\usepackage[polish]{babel}
%\usepackage{a4wide} \\węższe marginesy
\usepackage{array}
\usepackage{hhline}
\usepackage{booktabs}
%\usepackage{forloop}
%\usepackage{pgffor}
\usepackage{fancyhdr}
\usepackage{multirow}
\usepackage[pdftex]{graphicx}
\usepackage{enumerate} %[I], numerki, [(a)]
%ustawienie poziomów wypunktowania do wyboru: $\bullet$, $\cdot$, $\diamond$, $-$, $\ast$ and $\circ$ 
\renewcommand{\labelitemi}{$\diamond$}
\renewcommand{\labelitemii}{$\bullet$}
\renewcommand{\labelitemiii}{$-$}
\renewcommand{\labelitemiv}{$\ast$}

\usepackage[colorlinks=true,urlcolor=blue,linkcolor=red,citecolor=green]{hyperref}
\bibliographystyle{plain}

%marginesy - jeszcze weższe
%\usepackage{geometry}
%\newgeometry{tmargin=2cm, bmargin=2cm, lmargin=1cm, rmargin=1cm}

%Uwaga - testuję nowe komendy
\renewcommand{\[}{\begin{equation}}
\renewcommand{\]}{\end{equation}}

%tabelki
\usepackage{tabularx}

\author{Jakub Nowak, Żaneta Błaszczuk, Filip Kubicz, Jakub Porębski, Rafał Kozik}
%\title{MODELOWANIE}        

\begin{document}
\pagestyle{fancy}
\fancyhead[R]{Automatyka i robotyka rok II}
\fancyhead[L]{Modelowanie układów fizyczno - biologicznych}     
\fancyfoot[C]{\thepage}
\input{./titlepage.tex}
%
\begin{chapter}
\newpage
\section{Testowanie Generatorów liczb losowych i pseudolosowych}
 sadd
\end{chapter}
 
%polskie znaki
%ę ą ś ć ń ó ł ź ż

%testy
\\
%$f(x) = \left\{ x \atop y+2 \right. $
%\begin{displaymath}
%y = \left\{ \begin{array}{ll}
%a & \textrm{jezeli $d>c$}\\
%b+x & \textrm{rano}\\
%l & \textrm{w~ciagu dnia}
%\end{array} \right.
%\end{displaymath}
%$ $\\
%\input{./zadanie2.tex}
%\begin{titlepage}

\newcommand{\HRule}{\rule{\linewidth}{0.5mm}} % Defines a new command for the horizontal lines, change thickness here

\center % Center everything on the page
 
%----------------------------------------------------------------------------------------
%	HEADING SECTIONS
%----------------------------------------------------------------------------------------

\textsc{\LARGE Akademia Górniczo - Hutnicza im. Stanisława Staszica}\\[1.5cm] % Name of your university/college
\textsc{\Large Wydział Elektrotechniki, Automatyki, Informatyki i Inżynierii Biomedycznej}\\[0.5cm] % Major heading such as course name
\textsc{\large Kierunek: Automatyka i robotyka}\\[0.5cm] % Minor heading such as course title

%----------------------------------------------------------------------------------------
%	TITLE SECTION
%----------------------------------------------------------------------------------------

\HRule \\[0.4cm]
{ \huge \bfseries Modelowanie układów fizyczno - biologicznych}\\[0.4cm] % Title of your document
\HRule \\[1.5cm]
 
%----------------------------------------------------------------------------------------
%	AUTHOR SECTION
%----------------------------------------------------------------------------------------
%
\begin{minipage}{0.4\textwidth}
\begin{flushleft} \large
\emph{Autorzy:}\\
Jakub \textsc{Nowak},\\ 
Żaneta \textsc{Błaszczuk},\\ 
Filip \textsc{Kubicz}, \\
Jakub \textsc{Porębski}, \\
Rafał \textsc{Kozik}% Your name
\end{flushleft}
\end{minipage}
~
\begin{minipage}{0.4\textwidth}
\begin{flushright} \large
\emph{Prowadzący:} \\
Dr  \textsc{Wochlik} % Supervisor's Name
\end{flushright}
\end{minipage}\\[4cm]


% If you don't want a supervisor, uncomment the two lines below and remove the section above
%\Large \emph{Prowadzący:}\\
%dr Andrzej \textsc{Adamski}\\[3cm] % Your name

%----------------------------------------------------------------------------------------
%	DATE SECTION
%----------------------------------------------------------------------------------------

%{\large \today}\\[3cm] % Date, change the \today to a set date if you want to be precise

%----------------------------------------------------------------------------------------
%	LOGO SECTION
%----------------------------------------------------------------------------------------

%\includegraphics[height=70mm]{icon.png}%\\[1cm] % Include a department/university logo - this will require the graphicx package
 
%----------------------------------------------------------------------------------------

\vfill % Fill the rest of the page with whitespace

\end{titlepage}

\section{Testowanie generatorów liczb losowych i pseudolosowych}
\subsection{Generatory}

\subsubsection{Generator losowy 1}
Proponuję linuksowy /dev/random --- brzmi ciekawie :)
\subsubsection{Generator pseudolosowy 2}
Wybierzmy coś ciekawego z wielkiej puli generatorów
\subsubsection{Generator liczb losowych - rozpad promieniotwórczy}
Co w rozpadzie promieniotwórczym jest elementem losowym?\\
$\Delta t$ okresu pomiędzy kolejnymi rozpadami?\\
Bo według mnie liczność zliczeń w danej sekundzie powinna się utrzymywać w okolicach stałej wartości, bo to jest własność rozpadu promieniotwórczego



\subsection{Testy}
\subsubsection{Test 1}
Wyboru testu dokonamy po przeanalizowaniu wbudowanych funkcji języka R i pakietu DieHarder
\subsubsection{Test 2}
\subsubsection{Test 3}


\section{Modelowanie stałoczasowego oraz inteligentnie sterowanego regulatora świateł drogowych}


\section{Modelowanie procesu ofiar i drapieżników}



\end{document}

%polskie znaki
%ę ą ś ć ń ó ł ź ż

%testy
\\
%$f(x) = \left\{ x \atop y+2 \right. $
%\begin{displaymath}
%y = \left\{ \begin{array}{ll}
%a & \textrm{jezeli $d>c$}\\
%b+x & \textrm{rano}\\
%l & \textrm{w~ciagu dnia}
%\end{array} \right.
%\end{displaymath}
%$ $\\

