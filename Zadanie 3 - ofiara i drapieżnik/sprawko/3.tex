\documentclass[11pt]{aghdpl}
% \documentclass[en,11pt]{aghdpl}  % praca w języku angielskim
\usepackage[polish]{babel}
%\usepackage[english]{babel}
\usepackage[utf8]{inputenc}
\usepackage[T1]{fontenc}

% dodatkowe pakiety
\usepackage{enumerate}
\usepackage{listings}
\lstloadlanguages{TeX}

\usepackage{listings}
\usepackage{xcolor}
\definecolor{listinggray}{gray}{0.9}
\definecolor{lbcolor}{rgb}{0.9,0.9,0.9}
\lstset{
    backgroundcolor=\color{lbcolor},
    tabsize=4,
  language=C++,
  captionpos=b,
  tabsize=3,
  frame=lines,
  numbers=left,
  numberstyle=\tiny,
  numbersep=5pt,
  breaklines=true,
  showstringspaces=false,
  basicstyle=\footnotesize,
%  identifierstyle=\color{magenta},
  keywordstyle=\color[rgb]{0,0,1},
  commentstyle=\color{Darkgreen},
  stringstyle=\color{red}
  }
\lstset{
  literate={ą}{{\k{a}}}1
           {ć}{{\'c}}1
           {ę}{{\k{e}}}1
           {ó}{{\'o}}1
           {ń}{{\'n}}1
           {ł}{{\l{}}}1
           {ś}{{\'s}}1
           {ź}{{\'z}}1
           {ż}{{\.z}}1
           {Ą}{{\k{A}}}1
           {Ć}{{\'C}}1
           {Ę}{{\k{E}}}1
           {Ó}{{\'O}}1
           {Ń}{{\'N}}1
           {Ł}{{\L{}}}1
           {Ś}{{\'S}}1
           {Ź}{{\'Z}}1
           {Ż}{{\.Z}}1
}



%\usepackage[pdftex]{graphicx}
\usepackage{tikz}
%---------------------------------------------------------------------------

\author{Żaneta Błaszczuk, Rafał Kozik, Filip Kubicz, Jakub Nowak, Jakub Porębski}
\shortauthor{Ż. Błaszczuk, R. Kozik, F. Kubicz, J. Nowak, J. Porębski}

\titlePL{Podwójne wachadło}
\titleEN{Double pendulum}

\shorttitlePL{Podwójne wachadło} % skrócona wersja tytułu jeśli jest bardzo długi
%\shorttitleEN{Thesis in \LaTeX}

\thesistype{Modelowanie układów fizycznych i biologicznych}
%\thesistype{Master of Science Thesis}

\supervisor{dr }

\degreeprogramme{Automatyka i Robotyka}
%\degreeprogramme{Computer Science}

\date{2014}

\department{Katedra Automtyki}
%\department{Department of Applied Computer Science}

\faculty{Wydział Elektrotechniki, Automatyki,\protect\\[-1mm] Informatyki i Inżynierii Biomedycznej}
%\faculty{Faculty of Electrical Engineering, Automatics, Computer Science and Biomedical Engineering}

\setlength{\cftsecnumwidth}{10mm}

\begin{document}
\titlepages

\section{Teoria}
Wachadło podwójne jest to wachadło matematyczne zawieszone na drugim wachadle matematycznym. Jego schemat pokazuje rys. \ref{Schemat}.
\begin{figure}[h!]
\centering
\label{Schemat}
\setlength{\unitlength}{2mm}
\begin{picture}(50,35)(0,15)

\put(25,50){\vector(1,0){3}}
\put(25,50){\vector(0,-1){3}}
\put(27,49){\makebox(0,0){$x$}}
\put(24,48){\makebox(0,0){$y$}}

\put(20,50){\line(1,0){10}}
\multiput(25,48)(0,-2){7}{\line(0,1){1}}
\put(25,50){\line(1,-3){5}}
\put(30,35){\circle*{2}}

\put(27,38){\makebox(0,0){$\varphi_1$}}
\put(30,40){\makebox(0,0){$l_1$}}
\put(33,35){\makebox(0,0){$m_1$}}

\multiput(30,33)(0,-2){7}{\line(0,1){1}}
\put(30,35){\line(1,-5){4}}
\put(34,15){\circle*{2}}

\put(31.5,22){\makebox(0,0){$\varphi_2$}}
\put(33,25){\makebox(0,0){$l_2$}}
\put(37,15){\makebox(0,0){$m_2$}}

\end{picture}
\caption{Schemat wachadła}
\end{figure}
Wachadło opisuje 5 parametrów masy $m_1$ i $m_2$, długości $l_1$ i $l_2$ oraz przyśpieszenie ziemskie $g$. Można jednka zmniejszyś ich ilość podstawiając: 
\begin{equation}
	A = \frac{m_1}{m_2} \qquad B = \frac{l_2}{l_1} \qquad C = \frac{g}{l_1}
\end{equation}
Stan wachadł opisują cztery parametry: kąty $\varphi_1$ i $\varphi_2$ oraz prędkości kątowe $\omega_1$ i $\omega_2$. 
Ruch opisuje układ czterech równań różniczkowych:
\begin{equation}
	\dot{\varphi}_1 = \omega_1
\end{equation}
\begin{equation}
	\dot{\omega}_1=-\frac{\sin(\varphi_1-\varphi_2)(B\omega_2^2+\omega_1^2cos(\varphi_1-\varphi_2))+C((A+1)\sin(\varphi_1)-
	\sin(\varphi_2)cos(\varphi_1-\varphi_2))}{A+\sin^2(\varphi_1-\varphi_2)}
\end{equation}
\begin{equation}
	\dot{\varphi}_2 = \omega_2
\end{equation}
\begin{equation}
	\dot{\omega}_2 = \frac{(A+1)(\omega_1^2\sin(\varphi_1-\varphi_2)-C\sin(p2))+\cos(\varphi_1-\varphi_2)((B\omega_2^2 \sin(\varphi_1-\varphi_2))+C(A+1)\sin(\varphi_1))}{B(A+\sin^2(\varphi_1-\varphi_2)}
\end{equation}
W modelu nie uwzględniono tarcia.
\section{Implementacja}
Równiania zostały rozwiązane numerycznie za pomocą metody Rungego-Kutty czwartego rzędu która została zaimplementowana w języku c++:
\begin{lstlisting}
  void next_step(
    e1 = next_e1();
    e2 = next_e2();

    float k1_1 = h*n_e1(p1,p2,w1,w2);
    float k1_2 = h*n_e2(p1,p2,w1,w2);
    float k1_3 = h*(w1);
    float k1_4 = h*(w2);

    float k2_1 = h*n_e1(p1+k1_3/2, p2+k1_4/2, w1+k1_1/2, w2+k1_2/2);
    float k2_2 = h*n_e2(p1+k1_3/2, p2+k1_4/2, w1+k1_1/2, w2+k1_2/2);
    float k2_3 = h*(k1_1/2+w1);
    float k2_4 = h*(k1_2/2+w2);

    float k3_1 = h*n_e1(p1+k2_3/2, p2+k2_4/2, w1+k2_1/2, w2+k2_2/2);
    float k3_2 = h*n_e2(p1+k2_3/2, p2+k2_4/2, w1+k2_1/2, w2+k2_2/2);
    float k3_3 = h*(k2_1/2+w1);
    float k3_4 = h*(k2_2/2+w2);

    float k4_1 = h*n_e1(p1+k3_3, p2+k3_4, w1+k3_1, w2+k3_2);
    float k4_2 = h*n_e2(p1+k3_3, p2+k3_4, w1+k3_1, w2+k3_2);
    float k4_3 = h*(k3_1+w1);
    float k4_4 = h*(k3_2+w2);

    w1 += (k1_1+2*k2_1+2*k3_1+k4_1)/6;
    w2 += (k1_2+2*k2_2+2*k3_2+k4_2)/6;
    p1 += (k1_3+2*k2_3+2*k3_3+k4_3)/6;
    p2 += (k1_4+2*k2_4+2*k3_4+k4_4)/6;

    i++;
  }


\end{lstlisting}

\section{Wyniki}






\section{Bibliografia}
1. Wróblewski J. praca licencjacka Wahadło podwójne Warszawa 2011

\end{document}